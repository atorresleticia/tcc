\chapter{Introdução}
\label{intro}

Vivencia-se atualmente uma economia caracterizada por uma rápida e constante mudança de mercado e oportunidades de negócio, 
tal que se tornou essencial às empresas a tomada de decisão correta e de forma rápida baseada em alguma decisão de negócio. 
Essas decisões são tomadas com base na análise de situações passadas e presentes de uma empresa, ou seja, com base nos dados armazenados por ela. 
Além disso, uma boa decisão também se utiliza da análise de mercado e predições.

Sob a ótica operacional, de acordo com Wremble e Koncilia \cite{wrembel2007data}, os dados de uma empresa são persistidos em sistemas de armazenamento 
de dados que podem ser heterogêneos, autônomos, e geograficamente distribuídos. Estas características diminuem a eficiência no acesso e processamento dos dados. 
A gestão de uma empresa requer, no entanto, uma visão abrangente de todos os seus aspectos, exigindo acesso eficiente a todos os dados de interesse. 
Por este motivo, a capacidade de integrar informações de várias fontes de dados é crucial para uma boa decisão de negócios \cite{wrembel2007data}.

Para se obter êxito em uma decisão de negócio é trivial (i) recuperar os dados mantidos por uma empresa; e sobretudo 
(ii) desenvolver um ambiente de análise cujo objetivo deve ser não apenas informar o significado destes dados, 
mas sim especular cenários sobre eles com questionamentos como \textit{"e se"} ou \textit{"por quê?"} \cite{codd1998providing}. 
Segundo Chaudhuri e Dayal \cite{chaudhuri1997overview} dois elementos são essenciais, dadas as condições anteriores, para uma boa decisão de negócio: 
\textit{Data Warehouses} (DWs), responsáveis pelo armazenamento homogêneo de dados oriundos de sistemas heterogêneos, e recuperação destes dados; 
e aplicações OLAP (\textit{On-Line Analytical Processing}).

De acordo com a literatura, existe uma série de princípios que devem ser seguidos ao projetar e implementar um ambiente OLAP dentro de um DW, destacando-se a rapidez com que os dados devem ser recuperados e processados no DW. Este princípio é considerado fundamental 
na maior parte da literatura referente à construção de ambientes OLAP, a exemplo de Codd; Codd e Salley \cite{codd1998providing}, Kimball e Ross \cite{kimball2002dw} 
e Wremble e Koncilia \cite{wrembel2007data}. Sendo assim, vários aspectos técnicos devem ser considerados sob diferentes pontos de vista. 
Políticas de \textit{cache} específicas para um servidor de estruturas utilizadas pela aplicação OLAP para armazenar e recuperar dados em memória 
e o SGBD (Sistemas Gerenciadores de Banco de Dados) utilizado para o gerenciamento do DW são exemplos de aspectos técnicos que também fazem parte de um ambiente OLAP, 
e que devem ser considerados. 

De acordo com Elmasri e Navathe \cite{navathe2011banco} SGBD são importantes para a manutenção 
e proteção de um banco de dados por um longo período; e também atuam no processo de recuperação de dados de um DW. Neste contexto, 
a escolha do SGBD no processo de desenvolvimento de um ambiente de análise para organizações com grande quantidade de dados e que utilizam ferramentas OLAP 
como auxílio na tomada de decisões é importante. 

Duas classes de SGBD podem ser utilizadas para realizar o gerenciamento de um DW: SGBD relacionais tradicionais orientados à linha e uma nova abordagem de SGBD 
NoSQL que são orientados à coluna, que fornecem um melhor desempenho à recuperação de dados \cite{good2017column}. 
Dada a importância na escolha do SGBD e as diferentes soluções apresentadas por cada um, torna-se útil o uso de um 
\textit{benchmark}\footnote{Ferramentas utilizadas para medir e validar o desempenho de alguma tecnologia sob condições de avaliação \cite{bouckaert2010benchmarking}} 
para a realização de uma análise entre os SGBD. Existe uma organização sem fins lucrativos fundada com o objetivo de definir padrões para avaliar o 
desempenho de transações e de bancos de dados com o uso de \textit{benchmarks}, o TPC (\textit{Transactional Processing Performance Council}) \cite{tpc2017page}. 
Esta organização é responsável pela criação de um \textit{benchmark} voltado para decisões de negócio, o TPC-H \cite{tpch2017page}.

Para refletir a realidade de uma empresa, o TPC-H propõe um ambiente de análise normalizado. Segundo Bax e Souza \cite{bax2003modelagem} 
existe uma discussão sobre a normalização e a denormalização dos dados de um DW. Algumas empresas utilizam um modelo normalizado e têm como 
justificativa a flexibilidade e a facilidade em situações de manutenção do DW. Por outro lado, existem empresas que utilizam modelos denormalizados 
e têm como justificativa o ganho de desempenho nas consultas, visto que um modelo denormalizado tende a diminuir o número de tabelas e, por consequência, 
as operações de junção entre tabelas. 

O objetivo deste trabalho é a realização de um estudo comparativo entre os SGBD PostgreSQL e MonetDB, o primeiro relacional orientado à linha e o segundo orientado à coluna, 
como gerenciadores de DWs em ambientes OLAP. A comparação será realizada utilizando-se o \textit{benchmark} TPC-H em sua proposta original, um modelo de banco normalizado; 
além de uma adaptação para um modelo DW denormalizado. Desta maneira, será possível avaliar os SGBD selecionados de acordo com o modelo do DW, não apenas no 
contexto original proposto pelo \textit{benchmark}. Para alcançar tal objetivo, é necessária a execução de uma série de tarefas. A primeira parte engloba o 
desenvolvimento dos dois ambientes de análise, o modelo padrão proposto pelo TPC-H e o modelo denormalizado adaptado do TPC-H. Após, são gerados os 
dados para popular os SGBD e as consultas para as questões de negócio propostas pelo TPC-H; então é feita a população dos dados gerados nos SGBD. 
A segunda parte consiste na execução das consultas para cada um dos ambientes de análise de acordo com a metodologia proposta pelo TPC-H, sendo que para o ambiente adaptado as consultas devem 
ser escritas de acordo com as alterações efetuadas. E por fim a aplicação do cálculo de desempenho proposto pelo TPC-H para os dois ambientes. 

A organização do documento segue da seguinte forma:

\begin{itemize}
	\item \textbf{Capítulo 2}: apresenta detalhes sobre recuperação de informações, explicando os conceitos de \textit{Data Warehouse}, 
	ambientes OLAP e a conexão entre ambos.
	\item \textbf{Capítulo 3}: apresenta uma breve descrição da motivação no uso de SGBD como gerenciadores de DWs, bem como a descrição 
	do modelo relacional, suas limitações e uma introdução a NoSQL.
	\item \textbf{Capítulo 4}: apresenta detalhes de como e porquê surgiram os SGBD NoSQL, assim como a descrição do modelo colunar.
	\item \textbf{Capítulo 5}: apresenta detalhes sobre o \textit{benchmark} TPC-H, abrangendo informações sobre o ambiente normalizado proposto pelo próprio \textit{benchmark}; bem como sobre o ambiente denormalizado proposto neste trabalho. É também apresentada a metodologia utilizada para medição e análise do desempenho dos SGBD.
	\item \textbf{Capítulo 6}: é apresentado o estudo comparativo entre os SGBD PostgreSQL e MonetDB, dividido em cenários conforme o fator 
	de escala, bem como a discussão dos resultados obtidos.
	\item \textbf{Capítulo 7}: são discorridas as conclusões obtidas a partir dos resultados obtidos no experimento, assim como 
	trabalhos futuros.
\end{itemize}


