%============================================================

\chapter{Consultas do Ambiente Original Normalizado}
\label{queries_1}


São apresentadas aqui as 15 consultas realizadas no Ambiente \textit{snowflake}, cada qual com sua questão de negócio brevemente explicada.

\begin{enumerate}

%--------------------------------------1
% \item Consulta de Relatório de Resumo de Preços \\
% 	Emite um relatório com o resumo de preços de todos os itens enviados em uma certa data. Esta data corresponde a 60-120 dias antes da maior data de envio contida no banco de dados. 
    
% \begin{tabular}{ll}
% 	Parâmetro de Substituição & Valor\\
% 	DELTA & 90\\
% \end{tabular}
    
% 	\begin{lstlisting}[language=SQL]
% select
% 	l_returnflag,
% 	l_linestatus,
% 	sum(l_quantity) as sum_qty,
% 	sum(l_extendedprice) as sum_base_price,
% 	sum(l_extendedprice * (1 - l_discount)) as sum_disc_price,
% 	sum(l_extendedprice * (1 - l_discount) * (1 + l_tax)) as sum_charge,
% 	avg(l_quantity) as avg_qty,
% 	avg(l_extendedprice) as avg_price,
% 	avg(l_discount) as avg_disc,
% 	count(*) as count_order
% from
% 	lineitem
% where
% 	l_shipdate <= date '1998-12-01' - interval '[DELTA]' day (3)
% group by
% 	l_returnflag,
% 	l_linestatus
% order by
% 	l_returnflag,
% 	l_linestatus;
	
% 	\end{lstlisting}

% %--------------------------------------2
% \item Consulta de Fornecedor de Custo Mínimo

% 	Encontra o fornecedor que pode fornecer um produto com o menor custo em uma dada região.
	
% \begin{tabular}{ll}
% 	Parâmetro de Substituição & Valor\\
% 	SIZE & 15\\
% 	TYPE & BRASS\\
% 	REGION & EUROPE\\
% \end{tabular}

% 	\begin{lstlisting}[language=SQL]
% select
% 	s_acctbal,
% 	s_name,
% 	n_name,
% 	p_partkey,
% 	p_mfgr,
% 	s_address,
% 	s_phone,
% 	s_comment
% from
% 	part,
% 	supplier,
% 	partsupp,
% 	nation,
% 	region
% where
% 	p_partkey = ps_partkey
% 	and s_suppkey = ps_suppkey
% 	and p_size = [SIZE]
% 	and p_type like '%[TYPE]'
% 	and s_nationkey = n_nationkey
% 	and n_regionkey = r_regionkey
% 	and r_name = '[REGION]'
% 	and ps_supplycost = (
% 		select
% 			min(ps_supplycost)
% 		from
% 			partsupp,
% 			supplier,
% 			nation,
% 			region
% 		where
% 			p_partkey = ps_partkey
% 			and s_suppkey = ps_suppkey
% 			and s_nationkey = n_nationkey
% 			and n_regionkey = r_regionkey
% 			and r_name = '[REGION]'
% 	)
% order by
% 	s_acctbal desc,
% 	n_name,
% 	s_name,
% 	p_partkey;
% set rowcount 100
% go
% 	\end{lstlisting}

%--------------------------------------3
\item Consulta de Prioridade de Envio
	
	Retorna a prioridade de envio dos pedidos com a maior receita entre aqueles que ainda não foram enviados em uma determinada data.
	
\begin{tabular}{ll}
	Parâmetro de Substituição & Valor\\
	SEGMENT & BUILDING\\
	DATE & 1995-03-15\\
\end{tabular}

	\begin{lstlisting}[language=SQL]
select
	l_orderkey,
	sum(l_extendedprice * (1 - l_discount)) as revenue,
	o_orderdate,
	o_shippriority
from
	customer,
	orders,
	lineitem
where
	c_mktsegment = '[SEGMENT]'
	and c_custkey = o_custkey
	and l_orderkey = o_orderkey
	and o_orderdate < date '[DATE]'
	and l_shipdate > date '[DATE]'
group by
	l_orderkey,
	o_orderdate,
	o_shippriority
order by
	revenue desc,
	o_orderdate;
set rowcount 10
go
	\end{lstlisting}
	
%--------------------------------------4
% \item Consulta de Verificação de Prioridade de Pedido
		
% 	Faz a contagem do número de pedidos solicitados em um dado trimestre de um determinado ano no qual pelo menos um item foi recebido pelo cliente após a data de entrega.
	
% \begin{tabular}{ll}
% 	Parâmetro de Substituição & Valor\\
% 	DATE & 1993-07-01\\
% \end{tabular}

% 	\begin{lstlisting}[language=SQL]
% select
% 	o_orderpriority,
% 	count(*) as order_count
% from
% 	orders
% where
% 	o_orderdate >= date '[DATE]'
% 	and o_orderdate < date '[DATE]' + interval '3' month
% 	and exists (
% 		select
% 			*
% 		from
% 			lineitem
% 		where
% 			l_orderkey = o_orderkey
% 			and l_commitdate < l_receiptdate
% 	)
% group by
% 	o_orderpriority
% order by
% 	o_orderpriority;
	
% 	\end{lstlisting}

%--------------------------------------5
\item Consulta de Volume do Fornecedor Local
	
	Lista para cada nação em uma dada região o volume de receita originado de uma transação na qual cliente e fornecedor eram daquela nação.
	
\begin{tabular}{ll}
	Parâmetro de Substituição & Valor\\
	REGION & ASIA\\
	DATE & 1994-01-01\\
\end{tabular}

	\begin{lstlisting}[language=SQL]
select
	n_name,
	sum(l_extendedprice * (1 - l_discount)) as revenue
from
	customer,
	orders,
	lineitem,
	supplier,
	nation,
	region
where
	c_custkey = o_custkey
	and l_orderkey = o_orderkey
	and l_suppkey = s_suppkey
	and c_nationkey = s_nationkey
	and s_nationkey = n_nationkey
	and n_regionkey = r_regionkey
	and r_name = '[REGION]'
	and o_orderdate >= date '[DATE]'
	and o_orderdate < date '[DATE]' + interval '1' year
group by
	n_name
order by
	revenue desc;
	
	\end{lstlisting}
	
%--------------------------------------6
\item Previsão de Mudança de Receita
	
	Informa o quanto a receita pode aumentar eliminando alguns descontos em um dado ano.
	
\begin{tabular}{ll}
	Parâmetro de Substituição & Valor\\
	DATE & 1994-01-01\\
	DISCOUNT & .06\\
	QUANTITY & 24\\
\end{tabular}

	\begin{lstlisting}[language=SQL]
select
	sum(l_extendedprice * l_discount) as revenue
from
	lineitem
where
	l_shipdate >= date '[DATE]'
	and l_shipdate < date '[DATE]' + interval '1' year
	and l_discount between [DISCOUNT] - 0.01 and [DISCOUNT] + 0.01
	and l_quantity < [QUANTITY];
	
	\end{lstlisting}
	
%--------------------------------------7
\item Consulta de Quantidade de Envio

    Determina o valor de bens enviados entre nações para auxiliar na renegociação de contratos de envio.
    
\begin{tabular}{ll}
	Parâmetro de Substituição & Valor\\
	NATION1 & FRANCE\\
	NATION2 & GERMANY\\
\end{tabular}

	\begin{lstlisting}[language=SQL]
select
	supp_nation,
	cust_nation,
	l_year,
	sum(volume) as revenue
from
	(
		select
			n1.n_name as supp_nation,
			n2.n_name as cust_nation,
			extract(year from l_shipdate) as l_year,
			l_extendedprice * (1 - l_discount) as volume
		from
			supplier,
			lineitem,
			orders,
			customer,
			nation n1,
			nation n2
		where
			s_suppkey = l_suppkey
			and o_orderkey = l_orderkey
			and c_custkey = o_custkey
			and s_nationkey = n1.n_nationkey
			and c_nationkey = n2.n_nationkey
			and (
				(n1.n_name = '[NATION1]' and n2.n_name = '[NATION2]')
				or (n1.n_name = '[NATION2]' and n2.n_name = '[NATION1]')
			)
			and l_shipdate between date '1995-01-01' and date '1996-12-31'
	) as shipping
group by
	supp_nation,
	cust_nation,
	l_year
order by
	supp_nation,
	cust_nation,
	l_year;
	
	\end{lstlisting}

%--------------------------------------8
\item Consulta de Quota de Mercado Nacional

    Determina quanto a quota de mercado de uma dada nação em uma dada região mudou em dois anos para um determinado tipo de peça.
    
\begin{tabular}{ll}
	Parâmetro de Substituição & Valor\\
	NATION & BRAZIL\\
	REGION & AMERICA\\
	TYPE &  ECONOMY ANODIZED STEEL\\
\end{tabular}

	\begin{lstlisting}[language=SQL]
select
	o_year,
	sum(case
		when nation = '[NATION]' 
		then volume
		else 0
	end) / sum(volume) as mkt_share
from
	(
		select
			extract(year from o_orderdate) as o_year,
			l_extendedprice * (1 - l_discount) as volume,
			n2.n_name as nation
		from
			part,
			supplier,
			lineitem,
			orders,
			customer,
			nation n1,
			nation n2,
			region
		where
			p_partkey = l_partkey
			and s_suppkey = l_suppkey
			and l_orderkey = o_orderkey
			and o_custkey = c_custkey
			and c_nationkey = n1.n_nationkey
			and n1.n_regionkey = r_regionkey
			and r_name = '[REGION]'
			and s_nationkey = n2.n_nationkey
			and o_orderdate between date '1995-01-01' and date '1996-12-31'
						and p_type = '[TYPE]'
	) as all_nations
group by
	o_year
order by
	o_year;
	
	\end{lstlisting}

%--------------------------------------9
\item Consulta ao Lucro de um Tipo de Produto

    Encontra para cada nação em cada ano o lucro de todas as peças pedidas naquele ano contendo uma substring específica em seus nomes que foram preenchidos por um fornecedor naquela nação.
    
\begin{tabular}{ll}
	Parâmetro de Substituição & Valor\\
	COLOR & green\\
\end{tabular}

	\begin{lstlisting}[language=SQL]
select
	nation,
	o_year,
	sum(amount) as sum_profit
from
	(
		select
			n_name as nation,
			extract(year from o_orderdate) as o_year,
			l_extendedprice * (1 - l_discount) - ps_supplycost * l_quantity as a
mount
		from
			part,
			supplier,
			lineitem,
			partsupp,
			orders,
			nation
		where
			s_suppkey = l_suppkey
			and ps_suppkey = l_suppkey
			and ps_partkey = l_partkey
			and p_partkey = l_partkey
			and o_orderkey = l_orderkey
			and s_nationkey = n_nationkey
			and p_name like '%[COLOR]%'
	) as profit
group by
	nation,
	o_year
order by
	nation,
	o_year desc;
	
	\end{lstlisting}

%--------------------------------------10
\item Consulta de Relatório de Itens Retornados

    Retorna os 20 principais clientes que podem estar tendo problemas com peças que foram enviadas a eles em um dado trimestre.
    
\begin{tabular}{ll}
	Parâmetro de Substituição & Valor\\
	DATE & 1993-10-01\\
\end{tabular}

	\begin{lstlisting}[language=SQL]
select
	c_custkey,
	c_name,
	sum(l_extendedprice * (1 - l_discount)) as revenue,
	c_acctbal,
	n_name,
	c_address,
	c_phone,
	c_comment
from
	customer,
	orders,
	lineitem,
	nation
where
	c_custkey = o_custkey
	and l_orderkey = o_orderkey
	and o_orderdate >= date '[DATE]'
	and o_orderdate < date '[DATE]' + interval '3' month
	and l_returnflag = 'R'
	and c_nationkey = n_nationkey
group by
	c_custkey,
	c_name,
	c_acctbal,
	c_phone,
	n_name,
	c_address,
	c_comment
order by
	revenue desc;
set rowcount 20
go
	\end{lstlisting}

%--------------------------------------11
\item Consulta a Identificação de Estoques Importantes
    
    Avalia de todos os estoques de fornecedores disponíveis em uma dada nação as peças com uma percentagem significativa do total de peças disponíveis.
    
\begin{tabular}{ll}
	Parâmetro de Substituição & Valor\\
	NATION & GERMANY\\
	FRACTION & .0001\\
\end{tabular}

	\begin{lstlisting}[language=SQL]
select
	ps_partkey,
	sum(ps_supplycost * ps_availqty) as value
from
	partsupp,
	supplier,
	nation
where
	ps_suppkey = s_suppkey
	and s_nationkey = n_nationkey
	and n_name = '[NATION]'
group by
	ps_partkey having
		sum(ps_supplycost * ps_availqty) > (
			select
				sum(ps_supplycost * ps_availqty) * [FRACTION]
			from
				partsupp,
				supplier,
				nation
			where
				ps_suppkey = s_suppkey
				and s_nationkey = n_nationkey
				and n_name = '[NATION]'
		)
order by
	value desc;
	
	\end{lstlisting}

%--------------------------------------12
\item Consulta a Modos de Envio e Prioridade de Pedidos

    Determina quando selecionar modos de envio mais baratos afeta negativamente pedidos com alta prioridade.
    
\begin{tabular}{ll}
	Parâmetro de Substituição & Valor\\
	SHIPMODE1 & MAIL\\
	SHIPMODE2 & SHIP\\
	DATE & 1994-01-01 \\
\end{tabular}

	\begin{lstlisting}[language=SQL]
select
	l_shipmode,
	sum(case
		when o_orderpriority = '1-URGENT'
			or o_orderpriority = '2-HIGH'
			then 1
		else 0
	end) as high_line_count,
	sum(case
		when o_orderpriority <> '1-URGENT'
			and o_orderpriority <> '2-HIGH'
			then 1
		else 0
	end) as low_line_count
from
	orders,
	lineitem
where
	o_orderkey = l_orderkey
	and l_shipmode in ('[SHIPMODE1]', '[SHIPMODE2]')
	and l_commitdate < l_receiptdate
	and l_shipdate < l_commitdate
	and l_receiptdate >= date '[DATE]'
	and l_receiptdate < date '[DATE]' + interval '1' year
group by
	l_shipmode
order by
	l_shipmode;
	
	\end{lstlisting}
	
%--------------------------------------13
\item Consulta à Distribuição de Clientes

Determina a distribuição de clientes pelo número de pedidos que eles fizeram.

\begin{tabular}{ll}
	Parâmetro de Substituição & Valor\\
	WORD1 & special\\
	WORD2 & requests\\
\end{tabular}

	\begin{lstlisting}[language=SQL]
select
	c_count,
	count(*) as custdist
from
	(
		select
			c_custkey,
			count(o_orderkey)
		from
			customer left outer join orders on
				c_custkey = o_custkey
				and o_comment not like '%[WORD1]%[WORD2]%'
		group by
			c_custkey
	) as c_orders (c_custkey, c_count)
group by
	c_count
order by
	custdist desc,
	c_count desc;
	
	\end{lstlisting}

%--------------------------------------14
\item Consulta aos Efeitos de Promoção

Informa o retorno de mercado a uma propaganda, como um comercial de televisão ou uma campanha especial.

\begin{tabular}{ll}
	Parâmetro de Substituição & Valor\\
	DATE & 1995-09-01\\
\end{tabular}

	\begin{lstlisting}[language=SQL]
select
	100.00 * sum(case
		when p_type like 'PROMO%'
			then l_extendedprice * (1 - l_discount)
		else 0
	end) / sum(l_extendedprice * (1 - l_discount)) as promo_revenue
from
	lineitem,
	part
where
	l_partkey = p_partkey
	and l_shipdate >= date '[DATE]'
	and l_shipdate < date '[DATE]' + interval '1' month;
	
	\end{lstlisting}

%--------------------------------------15
% \item Consulta de Fornecedor Principal

% Encontra o fornecedor que mais contribuiu com a receita total de peças enviadas durante um dado trimestre de um ano.

% \begin{tabular}{ll}
% 	Parâmetro de Substituição & Valor\\
% 	DATE & 1996-01-01\\
% \end{tabular}

% 	\begin{lstlisting}[language=SQL]
% create view revenue[STREAM_ID] (supplier_no, total_revenue) as
% 	select
% 		l_suppkey,
% 		sum(l_extendedprice * (1 - l_discount))
% 	from
% 		lineitem
% 	where
% 		l_shipdate >= date '[DATE]'
% 		and l_shipdate < date '[DATE]' + interval '3' month
% 	group by
% 		l_suppkey;

% select
% 	s_suppkey,
% 	s_name,
% 	s_address,
% 	s_phone,
% 	total_revenue
% from
% 	supplier,
% 	revenue[STREAM_ID]
% where
% 	s_suppkey = supplier_no
% 	and total_revenue = (
% 		select
% 			max(total_revenue)
% 		from
% 			revenue[STREAM_ID]
% 	)
% order by
% 	s_suppkey;

% drop view revenue[STREAM_ID];

% 	\end{lstlisting}

%--------------------------------------16
\item Consulta à Relação \textit{Part/Supplier}

Descobre quantos fornecedores podem fornecer peças com determinados atributos requeridos por um cliente. 

\begin{tabular}{ll}
	Parâmetro de Substituição & Valor\\
	BRAND & Brand$\#$45\\
	TYPE & MEDIUM POLISHED \\
	SIZE1 & 49 \\
	SIZE2 & 14 \\
	SIZE3 & 23 \\
	SIZE4 & 45 \\
	SIZE5 & 19 \\
	SIZE6 &  3 \\
	SIZE7 & 36 \\
	SIZE8 &  9 \\
\end{tabular}

	\begin{lstlisting}[language=SQL]
select
	p_brand,
	p_type,
	p_size,
	count(distinct ps_suppkey) as supplier_cnt
from
	partsupp,
	part
where
	p_partkey = ps_partkey
	and p_brand <> '[BRAND]'
	and p_type not like '[TYPE]%'
	and p_size in ([SIZE1], [SIZE2], [SIZE3], [SIZE4], [SIZE5], [SIZE6], [SIZE7], [SIZE8])
	and ps_suppkey not in (
		select
			s_suppkey
		from
			supplier
		where
			s_comment like '%Customer%Complaints%'
	)
group by
	p_brand,
	p_type,
	p_size
order by
	supplier_cnt desc,
	p_brand,
	p_type,
	p_size;
	
	\end{lstlisting}

%--------------------------------------17
% \item Consulta a Receita de Pedidos de Pequenas Quantidades

% Determina quanto de receita seria perdido se não fossem mais feitos pedidos para quantidades pequenas de certas peças.

% \begin{tabular}{ll}
% 	Parâmetro de Substituição & Valor\\
% 	BRAND & Brand$\#$23\\
% 	TYPE & MEDIUM POLISHED \\
% 	CONTAINER & MED BOX \\
% \end{tabular}

% 	\begin{lstlisting}[language=SQL]
% select
% 	sum(l_extendedprice) / 7.0 as avg_yearly
% from
% 	lineitem,
% 	part
% where
% 	p_partkey = l_partkey
% 	and p_brand = '[BRAND]'
% 	and p_container = '[CONTAINER]'
% 	and l_quantity < (
% 		select
% 			0.2 * avg(l_quantity)
% 		from
% 			lineitem
% 		where
% 			l_partkey = p_partkey
% 	);
	
% 	\end{lstlisting}

%--------------------------------------18
\item Consulta a Grandes Volumes de Clientes

Classifica os 100 principais clientes que já realizaram grandes quantidades de pedidos. 

\begin{tabular}{ll}
	Parâmetro de Substituição & Valor\\
	QUANTITY & 300\\
\end{tabular}

	\begin{lstlisting}[language=SQL]
select
	c_name,
	c_custkey,
	o_orderkey,
	o_orderdate,
	o_totalprice,
	sum(l_quantity)
from
	customer,
	orders,
	lineitem
where
	o_orderkey in (
		select
			l_orderkey
		from
			lineitem
		group by
			l_orderkey having
				sum(l_quantity) > [QUANTITY]
	)
	and c_custkey = o_custkey
	and o_orderkey = l_orderkey
group by
	c_name,
	c_custkey,
	o_orderkey,
	o_orderdate,
	o_totalprice
order by
	o_totalprice desc,
	o_orderdate;
	
	\end{lstlisting}
	
%--------------------------------------19
\item Consulta a Desconto de Receita

Encontra o desconto bruto de todos os pedidos para três diferentes tipos de peças que foram enviadas por via aérea e entregues pessoalmente.

\begin{tabular}{ll}
	Parâmetro de Substituição & Valor\\
	QUANTITY1 & 300\\
	BRAND1 & Brand$\#12$ \\
	QUANTITY2 & 10 \\
	BRAND2 & Brand$\#23$ \\
	QUANTITY3 & 20 \\
	BRAND3 & Brand$\#34$ \\
\end{tabular}

	\begin{lstlisting}[language=SQL]
select
	sum(l_extendedprice* (1 - l_discount)) as revenue
from
	lineitem,
	part
where
	(
		p_partkey = l_partkey
		and p_brand = '[BRAND1]'
		and p_container in ('SM CASE', 'SM BOX', 'SM PACK', 'SM PKG')
		and l_quantity >= [QUANTITY1] and l_quantity <= [QUANTITY1] + 10
		and p_size between 1 and 5
		and l_shipmode in ('AIR', 'AIR REG')
		and l_shipinstruct = 'DELIVER IN PERSON'
	)
	or
	(
		p_partkey = l_partkey
		and p_brand = '[BRAND2]'
		and p_container in ('MED BAG', 'MED BOX', 'MED PKG', 'MED PACK')
		and l_quantity >= [QUANTITY2] and l_quantity <= [QUANTITY2] + 10
		and p_size between 1 and 10
		and l_shipmode in ('AIR', 'AIR REG')
		and l_shipinstruct = 'DELIVER IN PERSON'
	)
	or
	(
		p_partkey = l_partkey
		and p_brand = '[BRAND3]'
		and p_container in ('LG CASE', 'LG BOX', 'LG PACK', 'LG PKG')
		and l_quantity >= [QUANTITY3] and l_quantity <= [QUANTITY3] + 10
		and p_size between 1 and 15
		and l_shipmode in ('AIR', 'AIR REG')
		and l_shipinstruct = 'DELIVER IN PERSON'
	);
	
	\end{lstlisting}

%--------------------------------------20
% \item Consulta a Potenciais Promoções de Partes

% Identifica os fornecedores de uma nação que possuem uma dada peça em excesso, que podem ser candidatas a uma oferta promocional.

% \begin{tabular}{ll}
% 	Parâmetro de Substituição & Valor\\
% 	COLOR & forest\\
% 	DATE & 1994-01-01\\
% 	NATION & CANADA\\
% \end{tabular}

% 	\begin{lstlisting}[language=SQL]
% select
% 	s_name,
% 	s_address
% from
% 	supplier,
% 	nation
% where
% 	s_suppkey in (
% 	select
% 		ps_suppkey
% 	from
% 		partsupp
% 	where
% 		ps_partkey in (
% 			select
% 				p_partkey
% 			from
% 				part
% 			where
% 				p_name like '[COLOR]%'
% 		)
% 		and ps_availqty > (
% 			select
% 				0.5 * sum(l_quantity)
% 			from
% 				lineitem
% 			where
% 				l_partkey = ps_partkey
% 				and l_suppkey = ps_suppkey
% 				and l_shipdate >= date '[DATE]'
% 				and l_shipdate < date '[DATE]' + interval '1' year
% 		)
% 	)
% 	and s_nationkey = n_nationkey
% 	and n_name = '[NATION]'
% order by
% 	s_name;
	
% 	\end{lstlisting}

% %--------------------------------------21
% \item Consulta a Fornecedores que Mantiveram Pedidos em Espera

% Identifica fornecedores que não puderam enviar peças pedidas em um tempo hábil.

% \begin{tabular}{ll}
% 	Parâmetro de Substituição & Valor\\
% 	NATION & SAUDI ARABIA\\
% \end{tabular}

% 	\begin{lstlisting}[language=SQL]
% select
% 	s_name,
% 	count(*) as numwait
% from
% 	supplier,
% 	lineitem l1,
% 	orders,
% 	nation
% where
% 	s_suppkey = l1.l_suppkey
% 	and o_orderkey = l1.l_orderkey
% 	and o_orderstatus = 'F'
% 	and l1.l_receiptdate > l1.l_commitdate
% 	and exists (
% 		select
% 			*
% 		from
% 			lineitem l2
% 		where
% 			l2.l_orderkey = l1.l_orderkey
% 			and l2.l_suppkey <> l1.l_suppkey
% 	)
% 	and not exists (
% 		select
% 			*
% 		from
% 			lineitem l3
% 		where
% 			l3.l_orderkey = l1.l_orderkey
% 			and l3.l_suppkey <> l1.l_suppkey
% 			and l3.l_receiptdate > l3.l_commitdate
% 	)
% 	and s_nationkey = n_nationkey
% 	and n_name = '[NATION]'
% group by
% 	s_name
% order by
% 	numwait desc,
% 	s_name;
% set rowcount 100
% go
% 	\end{lstlisting}

%--------------------------------------22
\item Consulta a Oportunidades de Vendas Globais

Conta quantos clientes de determinados países não realizaram pedidos durante sete anos, porém têm chances de realizar um pedido.

\begin{tabular}{ll}
	Parâmetro de Substituição & Valor\\
	I1 & 13 \\
	I2 & 31 \\
	I3 & 23 \\
	I4 & 29 \\
	I5 & 30 \\
	I6 & 18 \\
	I7 & 17 \\
\end{tabular}

	\begin{lstlisting}[language=SQL]
select
	cntrycode,
	count(*) as numcust,
	sum(c_acctbal) as totacctbal
from
	(
		select
			substring(c_phone from 1 for 2) as cntrycode,
			c_acctbal
		from
			customer
		where
			substring(c_phone from 1 for 2) in
				('[I1]', '[I2]', '[I3]', '[I4]', '[I5]', '[I6]', '[I7]')
			and c_acctbal > (
				select
					avg(c_acctbal)
				from
					customer
				where
					c_acctbal > 0.00
					and substring(c_phone from 1 for 2) in
						('[I1]', '[I2]', '[I3]', '[I4]', '[I5]', '[I6]', '[I7]')
			)
			and not exists (
				select
					*
				from
					orders
				where
					o_custkey = c_custkey
			)
	) as custsale
group by
	cntrycode
order by
	cntrycode;
	
	\end{lstlisting}

\end{enumerate}